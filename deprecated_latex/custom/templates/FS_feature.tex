\newcommand{\useCase}[9]{
\h{4}{Use case}{
Based on p.98 (Praktisch UML, 5th edition)
}

\begin{tabu} to  \linewidth { X[.2cm] | X[l]}
  \hline
  Summary & #1 \\
  \hline
  Actor(s) & #2 \\
  \hline
  Precondition & #3 \\
  \hline
  Description & #4 \\
  \hline
  Intended flow & #5 \\ %how you predict it will happen
  \hline
  Alternative flow & #6 \\ %what could happen
  \hline
  Result & #7 \\
  \hline
\end{tabu}
\imgC{#8}{}{#9}{http://draw.io}
}


% 1,2,3= standard sentence, 4=How to Demo 5=estimated,6=spent time
\newcommand{\userStory}[6]{
\hfill $|$Estimated : #5 $|$ #6 : spent$|$ \\
As a(n) \textbf{#1},
I want to \textbf{#2},
so that \textbf{#3}.

\bigskip

\textbf{How to Demo}
#4
}

% 1=description, 2=imgSrc, 3=imgWidth
\newcommand{\domainModel}[3]{
\h{4}{Domain}{
\Isrc{https://en.wikipedia.org/wiki/Domain_model}{"The domain model is a representation of meaningful real-world concepts"}{oct '15}\\
Windesheim desires to have a domain model,
the best approach is to provide one for each user story.
This gives us a quick overview of the parts involved.
}
#1
\imgC{#2}{Domain model}{#3}{https://en.wikipedia.org/wiki/Domain_model}
}

\newcommand{\renderFSitem}{
\h{3}{\useCaseSummaryDATA}{}
\renderUserStory
\renderDomain
\renderUseCase \newpage
%
\emptyUserStory
\emptyDomainModel
\emptyUseCase
}




\newcommand{\renderUserStory}{
\userStory{\useCaseActorsDATA%\userStoryActorDATA%
}{\userStoryActionDATA%
}{\userStoryResultDATA%
}{\userStoryHowToDemoDATA%
}{\userStoryEstimatedTimeDATA%
}{\userStorySpentTimeDATA%
}
}

\newcommand{\emptyUserStory}{
\useCaseActors{}%\userStoryActor{}
\userStoryAction{}
\userStoryResult{}
\userStoryHowToDemo{}
\userStoryEstimatedTime{}
\userStorySpentTime{}
}

\newcommand{\renderUseCase}{
\useCase{\useCaseSummaryDATA%
}{\useCaseActorsDATA%
}{\useCasePreconditionDATA%
}{\useCaseDescriptionDATA%
}{\useCaseIntendedFlowDATA%
}{\useCaseAlternativeFlowDATA%
}{\userStoryResultDATA%\useCaseResultDATA%
}{\useCaseImgSrcDATA%
}{\useCaseImgWidthDATA%
}
}

\newcommand{\emptyUseCase}{
\useCaseSummary{}
\useCaseActors{}
\useCasePrecondition{}
\useCaseDescription{}
\useCaseIntendedFlow{}
\useCaseAlternativeFlow{}
\userStoryResult{}%\useCaseResult{}
\useCaseImgSrc{}
\useCaseImgWidth{}
}

\newcommand{\renderDomain}{
\domainModel{\domainDescriptionDATA%
}{\domainImgSrcDATA%
}{\domainImgWidthDATA%
}
}

\newcommand{\emptyDomainModel}{
\domainDescription{}
\domainImgSrc{}
\domainImgWidth{}
}
